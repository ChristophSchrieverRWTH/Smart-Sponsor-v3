\chapter{Introduction}
A majority of people consider donating to be an important activity. In the year 2018, 57\% of the population of the United Kingdom is quoted to have donated money to a charitable organization in any capacity \cite{giving19}. While this is a considerable number it is a reduction from the years before. This shows a continuation in the trend of declining donation numbers from the years 2017 and 2016 (60\% and 61\% respectively)\cite{giving19}. %This development was further observed in 2020 and 2021 but data from those years is discarded as it is skewed due to the impact of the COVID-19 pandemic \cite{giving20}\cite{giving21}.
\\
At the same time, there has been another trend of declining trust in charity organizations. While the number of people who say they trust charities in 2017 was 51\%, this number declined to 47\% in 2018. Similarly the number of people who actively distrust charity organizations increased from 19\% in 2017 to 21\% in 2018\cite{giving19}.\\
In a similar study by the UK's charity commission in 2016 trust in charity organizations on a numerical scale has dropped from 6.7 out of 10 in 2014 to 5.7 in 2016 \cite{trustgov}. This was a historical low since the start of monitoring in 2005. When asked for the main reasons which caused this erosion in trust, 32\% quoted media stories about how the donated money was spent. 21\% mentioned a lack of transparency about the monetary streams and 15\% of people complained about too much money spent on advertising/wages \cite{trustgov}. Pertaining to the last point 67\% of the surveyed people felt too much money was spent on administrative costs. This is an increase from the numbers of 2014 which made up 58\%.\\
Looking at these tendencies it seems important to increase both transparency and administrative efficiency in charity and sponsoring organizations to regain the public's trust. This is where the use of blockchains could help in furthering transparency and alleviating some of the administrative burden. The process of auditing a charity's use of funds also costs money. This problem can be lessened by blockchain-based systems. %The advent of blockchains and smart contracts allows more effective sponsoring, offering an alternative to conventional charities.
\\
There are already multiple donation systems entirely located on blockchain following different approaches. Some of them are focused on donating cryptocurrencies like Bitcoin \cite{bitgive}\cite{binance}. Others aim to reduce the strain on administrative efforts showing relations between charity organizations and related organizations. Thus allowing people to see if companies they don't want to support indirectly are involved \cite{disberse}. To address the complaint of money not reaching the projects it is supposed to help \cite{trustgov}, some systems release money after the charity has shown they put the money to good use \cite{alice}\cite{promisegive}.\\
These systems all have their strengths and weaknesses not just on a computational level, but also on a psychological level \cite{progDon}. Therefore further projects should be considered. The basis for the project in this paper is the concept of "Smart Money" \cite{weber}. Smart money describes tokens on a blockchain that are bound by certain conditions. The blockchain only allows a transaction to be completed if these conditions are fulfilled \cite{weber}. %These tokens are akin to vouchers or food stamps and have already had successful pilot projects in the social systems of Australia \cite{weber} and Finland \cite{kela}.
\\ 
The idea is to create a representation of fiat currency that can have spending conditions attached. An example could be "This donation can only be spent by a person who is younger than 22 and can only be spent at a store that sells books". As many conditions can be attached to a coin as is wished and only the owner of the coin can attach them. Using this baseline financial structure a system that allows the user to set up conditions attached to a donated amount can be created. A person who wants to receive this sponsorship then applies with this system after proving they fulfill these conditions. The system checks they actually fulfill the conditions and then transfers the donated amount, which subsequently can only be spent as long as all the conditions are fulfilled. Once the money is spent it returns to tokens without any conditions attached, allowing for normal use. For ease of reference, this project will be called "Smart Sponsor" from here on. \\
For this underlying infrastructure to work, there needs to be a trusted third party that verifies that certain users fulfill conditions e.g. "This store is a bookstore" or "This user is younger than 22". The process of verifying these conditions is assumed to happen off the blockchain and falls outside the scope of this project. This approach might be solved in a more distributed way or using an oracle \cite{pattern}. Either way, the implementation of the verifying authority is not the main focus of this paper.\\
After taking a look at the psychology behind donating and existing technologies in chapter 2, chapter 3 details the solution approach. Chapter 4 will then present the time plan for this project closing with a conclusion in chapter 5.