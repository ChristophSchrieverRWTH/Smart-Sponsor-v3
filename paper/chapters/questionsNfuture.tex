\chapter{Research Questions and Outlook}
This chapter will first expand on the research questions outlined in chapter 3. Following this, the outlook will detail possible expansions on the Smart Sponsor project or alternative solution approaches that could be the basis for follow-up projects.
\section{Research Questions}
This section will take a look at the questions posed at the end of chapter 3 and answer them with the experience gained from implementing Smart Sponsor.\\
\\
\emph{How can spending conditions be built into digital currency?}\\
Smart Sponsor implements conditions as strings with no semantic value. This allows for a broad implementation. This could be used further to use different string notations to mean different things. A possibility could be different delimiters denoting different boolean operands. Currently, everything is concatenated, but allowing for negation and conjunction cannot cause new issues while it could allow for more possibilities.\\
Another option could be the implementation of enumerations. Encoding a number to mean a certain string or a set of strings could allow faster confirmation of certificates. The drawback of this concept is the loss of security, as these numbers are more defined and therefore allow for a more methodical approach in brute-force testing.\\
\\
\emph{How can identity-related information be handled in a privacy-preserving manner?}\\
Smart Sponsors implementation assures the privacy of data by hashing over the information. This approach is not a Zero-Knowledge proof as the verifying entity holds the information. Nonetheless, this implementation gives an attacker only the option of brute force. Of course, the words that are contained in the string pose a threat to the security of the hashing algorithm. By including words into conditions the possible amount of input strings gets narrowed down significantly opening up for dictionary attacks.\\
As an added security measure it would be suggested to only allow certain smart contracts to check certificates that pertain to certificates that are not their own. In this way, a user can still ascertain they fulfill all their conditions while not giving any untrusted entities access to check their certificates.\\
\\
\emph{How can transparency in the donation-based sector be improved with blockchains?}\\
The blockchain offers two advantages to traditional currency regarding charitable giving. Firstly, it has an unchangeable log that is easily accessible. In this way, the movements of currency on a blockchain are traceable while fiat currency does not offer the same benefits, especially if in cash form.\\
Secondly, due to the attachment of Solidity as a programming language to the blockchain, it is possible to encode conditions as detailed in chapters 3 and 4. This allows for a proactive donation form that does not allow for the misappropriation of currency. Opposed stands fiat currency which is, according to the public perception, oftentimes not used properly\cite{giving19}\cite{trustgov}. Fiat currency is very much reactive with sanctions being enforced after misuse of currency has occurred. By this point the donor is dejected as his goal of helping has not been reached.\\
In these ways, the use of a blockchain can allow for securer and more reliable charitable giving with an easier tracing process.
\section{Outlook}
Smart Sponsor is a proof of concept. There are avenues to expand this project from both a technical perspective as well as a feature perspective.
\begin{itemize}
    \item Currently all certificates are stored at one third-party verification service. This poses an issue in two ways. Firstly, this is a single point of failure meaning if the verification service ceases operation, the entire system might break. Secondly, having only one service might lead to a lack of development due to a lack of competition. Allowing for more than one verification service might therefore open up more robustness and reap the benefits of competition.
    \item Another way to store certificates takes a step away from a third party holding this data. The concept of \emph{Distributed Identity} or DID for short allows users to hold on to their data\cite{DID}. Using DID the user would provide the data that is needed from their side, revealing only so much as they want to. These systems work in cooperation with blockchains and offer an opportunity to decentralise Smart Sponsor further, with all the benefits and ramifications that entails.
    \item If a larger focus is put on speed and less on the capability to trace the movement of singular pieces of currencies, changing the Sponsor Coins to something close to ERC-20 might allow for this. On the other hand, as currency would now exist as a number representing the current amount instead of separated tokens, the clear identity of the money donated would be lost.
    \item Giving semantic meaning to conditions might allow not only for improved performance but also for more possibilities regarding connected applications. Giving conditions semantic meaning might also allow for more meaningful extraction and aggregation of data.
    \item Currently currency in the system stays in the system. The creation of a withdrawal method incentivises stores and users to adopt this system. This might also be a needed measure to deal with inflation of the corresponding fiat currency.
    \item Smart Sponsor is currently designed to be used with singular users in mind. The system could be adapted to allow another use case supporting charity organisations with the same concepts of prescriptive conditions getting set by users. In these cases, it is probably for the best to allow the charity organisations to set condition templates the donors can use to not overwhelm the charity with the number of certificates they need to apply for.
    \item Developing an app might make this system more accessible as this can unite the steps of wallet creation and all steps MetaMask does into one place.
\end{itemize}