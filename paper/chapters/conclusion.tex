\chapter{Conclusion}
In this paper, the issue of a decrease in charitable giving to charities was analysed under the lens of trust. The constant decrease of trust in charity organisations plays a major part in this.\\
After analysing the motives for charitable giving, the current solutions to this problem are introduced. A point all these systems do not touch on is the concept of condition-based donations. Looking at condition-based currency in the field of healthcare, Smart Sponsor is modeled as a possible solution with the function of currency also being usable outside the donation context.\\
By creating a token currency that allows for conditions to be attached, a proactive system that does not allow misuse of money was designed.\\
The technology that makes this possible is the blockchain, as it allows for an immutable transaction history. This allows the users to trust a neutral system. Furthermore, the advent of smart contracts allows the creation of algorithms and data structures that can enforce the wishes of the blockchain's users.\\
Using Solidity and Ethereum, a proof of concept was developed. Running tests on this proof of concept, showed the functionality of the system, and using unit tests showed its correctness.\\
The cost of this prescriptive model lies in the performance of certain transactions. But as the large bulk of transactions will not attach conditions but just be simple movement of currency, this trade-off works to allow a flexible system.\\
With more refinement and further development, this system could be used as a platform to allow for more secure sponsorships and donations.